\documentclass{article}
\usepackage{verbatim}
\usepackage{moreverb}
\usepackage{fancyvrb}
\usepackage{graphics}
\usepackage{url}
\usepackage{mathbbol}
\usepackage{ifthen}
\usepackage[pdftex]{graphicx}
\let\verbatiminput=\verbatimtabinput
\def\etal{{et al.~}}
\title{A short user-guide to the {\tt ismt} tactic}
\author{
Levent Erk\"{o}k $\;\;\;\;$
John Matthews\\
\url{erkokl@gmail.com}\\
\url{matthews.r.john@gmail.com}\\$ $\\
}
\date{(First release: May 19th, 2008\\\vspace*{1mm}
Last updated: March 24th, 2011)}

\newboolean{submission} %set to true for the submission versio
\setboolean{submission}{false}
%\setboolean{submission}{true}
\ifthenelse
{\boolean{submission}}
{ \newcommand{\todo}[2]{ } } % hide todo
{ \newcommand{\todo}[2]{ % show todo
    \par
    \fbox{\parbox{0.7\columnwidth}{\textbf{#1}: #2}}
    \par               }}
\newcommand{\silent}[1]{} % comment-out

\begin{document}
\maketitle

\tableofcontents
\section{A note on which {\tt yices} to use}
The {\tt yices} executable that we predominantly use the {\tt ismt} tactic with must be in your path.
Note that the latest release of {\tt yices} (version 2.0) does {\em not} support
Yices's internal language, so you will need a 1-series release of {\tt yices} for {\tt ismt} to work. (We have tested
against version 1.0.29.) If you maintain multiple copies of {\tt yices} on your computer, then you can
use set the environment variable {\tt ISMT\_YICES} to tell the {\tt ismt} tactic precisely which {\tt yices}
executable to use. The following is an example setting, adjust the path accordingly on your platform:
\begin{verbatim}
  $ echo ${ISMT_YICES} 
  /usr/local/yices-1.0.29/bin/yices
\end{verbatim}
By using this environment variable you can have multiple versions of {\tt yices} happily coexisting on your computer,
letting {\tt ismt} use the correct version.
If this variable is not set, then {\tt ismt} will simply use the name {\tt yices}, which should be in your path
and must be a 1.X release.

\section{A note on Isabelle's oracle mechanism}
Prior to Isabelle-2009 releases, Isabelle used to tag theorems that are proved via
an oracle (such as {\tt ismt}) with the mark {\tt [!]}, clearly indicating that the
theorem is ``trusted.'' This has changed in recent versions of Isabelle, where Proof-General no longer displays this mark next to the theorems proved via oracles. This can give a false sense of security, in particular if you have used {\tt ismt} in the past with older versions of Isabelle where you do expect to see the trusted mark.
It turns out that Isabelle still keeps track of the oracle status, it's just a
matter of not displaying it directly. The reasoning behind this change is explained in the following
message:

\url{https://lists.cam.ac.uk/mailman/htdig/cl-isabelle-users/2009-April/msg00079.html}

In summary, one can still query the status of a theorem to ensure whether an oracle was
used in proving it (as it would be the case with {\tt ismt}), although Proof-General
no longer shows the mark {\tt [!]} to make it immediately obvious.
\section{Using the {\tt ismt} tactic} \label{sec:using}
The {\tt ismt} tactic has been designed to have a flexible user interface, allowing the user
to make various choices. It allows the following selections:
\begin{itemize}
\item {\em Solver backend:} Currently we only provide a backend translator for the Yices SMT solver, however the 
tactic is designed so that
other solvers can be plugged-in by providing an appropriate translator. 

\item {\em Choosing the model behavior:} The tactic can be instructed to either abort the proof or fail silently
when the underlying solver generates a satisfying model for the negation of the input.

\item {\em Saving the output file:} The user can instruct the {\tt ismt} tactic to save its output, mostly 
useful for debugging/inspection purposes.

\item {\em Turning on/off type-checking:} Some SMT solvers can be instructed to perform type-checking on
their input, or skip this step for efficiency purposes. The {\tt ismt} tactic allows the user to specify
the type-checking mode as an option.

\item {\em Turning on/off debug information:} If required, the {\tt ismt} tactic can be instructed to
produce a running narrative during translation, mainly useful for debugging/inspection purposes.

\item {\em Statistics reporting:} The {\tt ismt} tactic can provide run-time information on the translation
and backend components of the system, useful for benchmarking purposes.
\end{itemize}

In the following, we detail these parameters and show how they are used within the interactive 
theorem proving environment of Isabelle.

\subsection{Choosing the solver backend}
The back-end solver to use can be selected using the {\tt solver} flag:
\begin{itemize}
\item {\em Flag:} {\tt solver}
\item {\em Possible value:} {\tt Yices}
\item {\em Example:} 
\begin{verbatim}
    by (ismt solver: Yices)
\end{verbatim}
\item {\em Description:} Uses the specified solver as the underlying SMT solver. 
\item {\em Default:} If not specified, {\tt Yices} will be assumed.
\item {\em Remarks:} The {\tt yices} executable (downloadable on the web~\cite{YicesWWW}) should
be in user's path. Also see the note about {\tt ISMT\_YICES} environment variable above.
\end{itemize}
Note that this flag is currently redundant since Yices is the only backend 
we support for the time being.

\subsection{Choosing the model behavior}
This option controls how the tactic behaves if the underlying solver returns a satisfying assignment 
for the negation of the input, i.e., a counter-example. 
\begin{itemize}
\item {\em Flag:} {\tt model}
\item {\em Possible values:} {\tt silent}, {\tt notify}, {\tt abort}
\item {\em Examples:} 
\begin{verbatim}
    by (ismt model: silent)
    by (ismt model: abort)
\end{verbatim}
\item {\em Description:} In the {\tt silent} mode, the proof will fail by returning the 
Isabelle empty sequence {\tt Seq.empty}, with no further diagnostics. The
{\tt notify} mode is similar, except the counter-example will be printed in the {\tt *trace*} buffer of
Isabelle. If {\tt abort} is chosen, an exception will be thrown, failing the proof attempt.

The main use cases for the {\tt silent} and {\tt notify} modes are in combination with other tactics. 
For instance, a typical application of {\tt ismt} could be in combination with several other rewrite rules,
as in the following example:
\begin{verbatim}
by (  rule VC_rules 
    | simp only: VC_simps main_def Let_def 
    | ismt model: silent)+
\end{verbatim}
(with appropriate definitions of {\tt VC\_simps} etc.). In this case we apply one round
of rules and simplifications, followed by an attempt to prove by {\tt ismt}, and repeating the process until the 
goal is simplified enough such that {\tt ismt} can resolve it, or until none of the rules kick-in, hence causing the
proof to fail.
\item {\em Default:} If not specified, {\tt notify} will be assumed.
\end{itemize}

\subsection{Saving the output file}
For inspection and debugging purposes, the user might need to see the script generated by the 
{\tt ismt} tactic that is passed along to the underlying SMT solver.
This option allows specifying a file name for this output.
\begin{itemize}
\item {\em Flag:} {\tt dump}
\item {\em Possible values:} {\em Any valid file name, or {\tt -}.}
\item {\em Examples:} 
\begin{verbatim}
    by (ismt dump: "lemma.ys")
    by (ismt dump: -)
\end{verbatim}
\item {\em Description:} If a file name is given, then the script will be saved in that file. The 
character `{\tt -}' is
interpreted as {\tt stdout}, i.e., the script will be printed out directly on the screen. 
\item {\em Default:} If not specified, no dump file will be generated.
\item {\em Remarks:} The generated output will also contain the input HOL formula, the output of the SMT solver
when run, along with the counter-example translated back to HOL (if any). In this sense, it will contain enough
information to create a complete record of the transaction, which is an important aspect from an evaluator's
point-of-view. The generated file is also directly loadable by the underlying SMT solver, hence the corresponding
run can be independently repeated by the user outside of the Isabelle process.
\end{itemize}

\subsection{Turning on/off type-checking}
Some SMT solvers can be instructed to perform an extra step of type-checking on their input.
However, such a check can incur a performance penalty, so
these SMT solvers (and in particular Yices) also allow skipping this step for enhanced performance. The
{\tt tc\_on} and {\tt tc\_off} flags of the {\tt ismt} tactic allows passing this information down
to the underlying solver:
\begin{itemize}
\item {\em Flags:} {\tt tc\_on}, {\tt tc\_off}
\item {\em Examples:} 
\begin{verbatim}
    by (ismt tc_on)
    by (ismt tc_off)
\end{verbatim}
\item {\em Description:} If {\tt tc\_on} is specified, the backend solver will be instructed to perform
type-checking on the generated input. If {\tt tc\_off} is given, type-checking will be turned off.
\item {\em Default:} If not specified, {\tt tc\_on} is assumed.
\item {\em Remarks:} Unless efficiency is paramount, this flag should be left at its default value, i.e., {\tt tc\_on}.
In our test cases, we have found that the additional cost of type checking is practically negligible for the
Yices SMT solver.
\end{itemize}

\subsection{Debugging: Tracing {\tt ismt}}
The {\tt ismt} tactic provides a tracing mode, which is useful for debugging purposes. When
turned on, it will print out various run-time data. Users should typically 
avoid turning tracing on, as the
output tends to be quite copious especially with large input formulas.

\begin{itemize}
\item {\em Flags:} {\tt debug\_on}, {\tt debug\_off} 
\item {\em Examples:} 
\begin{verbatim}
    by (ismt debug_on)
    by (ismt debug_off)
\end{verbatim}
\item {\em Description:} Turns on/off tracing data output.
\item {\em Default:} If not specified, {\tt debug\_off} is assumed.
\end{itemize}

\subsection{Reporting statistics}
The {\tt ismt} tactic can report run-times of the translator itself and the backend SMT
solver, which
aids in benchmarking. This behavior is controlled by the {\tt stats\_on} and {\tt stats\_off} flags:
\begin{itemize}
\item {\em Flags:} {\tt stats\_on}, {\tt stats\_off}
\item {\em Examples:} 
\begin{verbatim}
    by (ismt stats_on)
    by (ismt stats_off)
\end{verbatim}
\item {\em Description:} The {\tt stats\_on} flag turns statistics reporting on, {\tt stats\_off}
turns it off. 
\item {\em Default:} If not specified, {\tt stats\_off} is assumed. 
\end{itemize}

\subsection{Argument order and defaults}

Multiple flags can be combined on the same line, or the same flag can be specified multiple times. In the latter
case, the last value given will be effective. The order of the given flags is immaterial. For instance, the following
calls are all valid:

\begin{verbatim}
    by (ismt model: silent debug_on dump: "a.ys" stats_on)
    by (ismt dump: "a.ys" dump: "b.ys")
\end{verbatim}

In the last example the {\tt dump} flag is repeated. The tactic
will use the second flag, hence the output will be placed in the file {\tt "b.ys"}.
It is also possible to call {\tt ismt} with no arguments at all:

\begin{verbatim}
    apply ismt
    by ismt
\end{verbatim}

\noindent In this case the defaults will be used for all the settings, as described
in the preceding sections.

\section{Supported constructs}
Below, we describe the types, constants, and other HOL constructs that are
properly understood and translated by the Yices backend. Any other construct will
go uninterpreted, i.e., it will be translated as an uninterpreted constant in the Yices
backend with the correct type.

\begin{itemize}
\item {\em Types.} The following types are supported:
\begin{itemize}
  \item Ground types: {\tt int}, {\tt nat}, {\tt bool}.
  \item Basic HOL types: (polymorphic) lists, option type, tuples (of arbitrary arity, including {\tt unit}).
  \item Records. (Extensible records are not supported.) 
  \item Types introduced via datatype declarations. These types can be parameterized
  and recursive. However, they cannot be mutually recursive, either directly or indirectly via nesting.
  \item Functions. Note that functions {\em can} be higher-order.
\end{itemize}
\item{\em Constants.}
The following HOL constants are supported:\footnote{The arithmetic operators ({\tt $+$}, {\tt $-$}, etc.), and
comparisons ({\tt <=}, {\tt <}) are supported
both at their {\tt int} and {\tt nat} instances. Use of arithmetic operators at
other Isabelle numeric types will remain uninterpreted. Also note that Yices only supports linear-arithmetic 
i.e., multiplication/division can only be done by a constant. If a non-linear expression is given to the
translator, it will still be translated but Yices will reject the input with a failure message.}
\begin{itemize}
  \item Equality: {\tt =}. (Polymorphic at all types.)
  \item Boolean operators: {\tt True}, {\tt False}, {\tt $\leq$}, {\tt $<$}, {\tt $\longrightarrow$}, {\tt $\Longrightarrow$}, 
  {\tt $\vee$}, {\tt $\wedge$}, {\tt $\neg$}, {\tt dvd}.
  \item Arithmetic operators:
       {\tt $+$}, {\tt $-$}, {\tt $\times$}, {\tt $/$}, 
       {\tt $-$} (unary minus), 
       {\tt div}, {\tt mod}, 
       {\tt abs}, {\tt Suc},
       {\tt min}, {\tt max}.
  \item Other: {\tt fst}, {\tt snd}.
\end{itemize} 
\item{\em Expressions and binding constructs.}
The following constructs are supported:
\begin{itemize}
  \item If expressions,
  \item Let bindings,
  \item Lambda abstractions,
  \item Quantifiers: $\forall$, $\exists$, $\bigwedge$ 
  \item Case expressions (over tuples, naturals, option type, lists, and arbitrary user defined types),
  \item Function update syntax,
  \item Record update syntax.
\end{itemize}
\end{itemize}
\newpage

\section{License}

\noindent Copyright \copyright 2007-2011. All rights reserved.

\noindent Redistribution and use in source and binary forms, with or without
modification, are permitted provided that the following conditions are
met:

\begin{itemize}
\item Redistributions of source code must retain the above copyright
notice, this list of conditions and the following disclaimer.

\item Redistributions in binary form must reproduce the above copyright
notice, this list of conditions and the following disclaimer in the
documentation and/or other materials provided with the distribution.

\item The name of the authors may not be used to  endorse or promote
products derived from this software without specific prior written
permission.
\end{itemize}

\noindent THIS SOFTWARE IS PROVIDED BY THE COPYRIGHT HOLDERS AND CONTRIBUTORS "AS
IS" AND ANY EXPRESS OR IMPLIED WAR\-RAN\-TIES, INCLUDING, BUT NOT LIMITED
TO, THE IMPLIED WARRANTIES OF MERCHANTABILITY AND FITNESS FOR A
PARTICULAR PURPOSE ARE DISCLAIMED. IN NO EVENT SHALL THE COPYRIGHT
OWNER OR CONTRIBUTORS BE LIABLE FOR ANY DIRECT, INDIRECT, INCIDENTAL,
SPECIAL, EXEMPLARY, OR CONSEQUENTIAL DAMAGES (INCLUDING, BUT NOT
LIMITED TO, PROCUREMENT OF SUBSTITUTE GOODS OR SERVICES; LOSS OF USE,
DATA, OR PROFITS; OR BUSINESS INTERRUPTION) HOWEVER CAUSED AND ON ANY
THEORY OF LIABILITY, WHETHER IN CONTRACT, STRICT LIABILITY, OR TORT
(INCLUDING NEGLIGENCE OR OTHERWISE) ARISING IN ANY WAY OUT OF THE USE
OF THIS SOFTWARE, EVEN IF ADVISED OF THE POSSIBILITY OF SUCH DAMAGE.
\bibliography{ismtUserGuide}
\bibliographystyle{plain}
\end{document}
